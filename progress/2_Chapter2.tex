%% ----------------------------------------------------------------
%% 2_Chapter2.tex
%% ---------------------------------------------------------------- 
\chapter{Background and Literature Review} \label{Chapter:two}

% Define your goal
% Do your research
% Ground summary in relevance
% Develop review logically
% Include references/works cited list

\section{Compliance}

    \subsection{What is Compliance?}
        Compliance generally refers to the conformance to a set of laws, regulations, policies, best practices, or service-level agreements. Compliance governance refers to the set of procedures, methodologies, and technologies put in place by a corporation to carry out, monitor, and manage compliance. Compliance governance is an important, expensive, and complex problem to deal with. (Silveira, P. et al. (2012) ‘Aiding Compliance Governance in Service-Based Business Processes’, in Handbook of Research on Service-Oriented Systems and Non-Functional Properties: Future Directions, pp. 524–548.)

    \subsection{Compliance in Cyber Security}
        Cybersecurity standards have existed over several decades as users and providers have collaborated in many domestic and international forums to effect the necessary capabilities, policies, and practices - generally emerging from work at the Stanford Consortium for Research on Information Security and Policy in the 1990s. (National Institute of Standards and Technology; Technology Administration; U.S. Department of Commerce., An Introduction to Computer Security: The NIST Handbook, Special Publication 800-12.)

\section{The State of Compliance in the UK: Cyber Essentials}
    \subsection{Cyber Essentials}
        The Government worked with the Information Assurance for Small and Medium Enterprises (IASME) consortium and the Information Security Forum (ISF) to develop Cyber Essentials, a set of basic technical controls to help organisations protect themselves against common online security threats. (Cyber Essentials Scheme: overview (2014) GOV.UK.)
    \subsection{Crime}
        We have seen a significant growth in cyber criminality in the form of high-profile ransomware campaigns over the last year. Breaches leaked personal data on a massive scale leaving victims vulnerable to fraud, while lives were put at risk and services damaged by the WannaCry ransomware campaign that affected the NHS and many other organisations worldwide. Tactics are currently shifting as businesses are targeted over individuals. (Cyber Crime (no date) NCA National Crime Agency.)

\section{The Impact of Security Breaches}
    \subsection{}
        Cyber attacks are financially devastating and disrupting and upsetting to people and businesses. (Cyber Crime (no date) NCA National Crime Agency.)

\section{Case Studies(?)}
