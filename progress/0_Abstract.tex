There are hundreds of cyber security compliance standards, and many businesses require their partners to comply with numerous standards. "Unlike cybersecurity alone, cyber supply chain risk management focuses on gaining visibility and control not only over the focal organization but also over its extended enterprise partners, such as Tier 1/Tier 2 suppliers and customers. In addition, while cybersecurity emphasizes purely technical means of control, CSCRM seeks to engage both managerial and human factors engineering in preventing risks from disrupting IT systems׳ operations."(Boyson, S. (2014) ‘Cyber supply chain risk management: Revolutionizing the strategic control of critical IT systems’, Technovation, 34(7), pp. 342\textendash 353.) Keeping track of each company’s compliance to a particular standard is a lengthy and potentially expensive task since it can be very difficult to maintain without the use of an external service or consultant. ("Says who?")

Most SMEs will not be able to afford this - due to the time and experience level required, it might not be something a system administrator can do on top of their other responsibilities, and a consultant might be too expensive.("Says who?")

An automatically generated cyber security compliance engine, could provide a low cost, time efficient solution for businesses that need a flexible, customisable way of tracking their partner’s compliance, or their own compliance, with multiple standards.("Says who?")

The goal of this project is to create a client-server system that will generate and store compliance forms for the end user. The forms will be automatically generated via an interface on the application by an ‘admin’, and accessible by ‘users’. This will include the ability to update the forms at a later date. This project is a client-server system only, not an application, and it will deal with cyber security compliance only - no other forms of compliance will be within the scope of this project.
